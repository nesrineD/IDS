%#############################################################
%###################### Statement ############################
%#############################################################
\chapter*{Plagiatserkl{\"a}rung}
%this one needs to be signed for submission
    Hiermit versichern wir, dass wir die vorliegende Arbeit \"{u}ber \textbf{Recent Research Trends of Artificial Immune Systems for IT-Security} selbst\"{a}ndig verfasst und nur die angegebenen Quellen und Hilfsmittel verwendet habe. Stellen der Arbeit, die anderen Werken -- auch elektronischen Medien -- dem Wortlaut oder Sinn nach entnommen wurden, habe ich unter Angabe der Quelle kenntlich gemacht.

\vspace{4cm}

\noindent Berlin, 29 Juli 2015 \hfill Signature

%#############################################################
%###################### German Abstract ######################
%#############################################################
%\newpage
%\chapter*{Zusammenfassung}
    
%    \textbf{Schlagw\"orter:}
%#############################################################
%###################### Abstract  ############################
%#############################################################
\newpage
\chapter*{Abstract}

    The field of intrusion detection has grown rapidly due to the growing concern of IT security.\ Perhaps the most important characteristic of the biological immune system is that it is particularly robust in defending the host against invading agents and at the same time especially impeccable in recognising and distinguishing between self and non self cells.\ By taking inspiration from the biological immune system, the artificial immune system (AIS) shows a promise to deliver solutions to the issue of IT security, especially in detecting intrusion.\ This paper introduces and describes three recent papers published in the field of AIS for IT security from the International Conference on Artificial Immune System (ICARIS) 2012.\ A subjective evaluation of each works is then presented in conclusion of this paper.
    
    \noindent\textbf{Keywords:} Artificial Immune System, AIS, IT-Security  
    
    
%DELETEME: An abstract is a teaser for your work. You try to convince a reader that it is worth reading your work. Normally, it makes to structure you abstract in this way: 
%\begin{itemize}
%\item one paragraph on the motivation to your topic
%\item one paragraph on what approach you have chosen
%\item and one paragraph on your results which may be presented in comparison to other approaches that try to solve the same or a similar problem.
%\end{itemize}
%Abstract should not exceed one page (aubrey's opinion)

%DELETEME: translate to German to Englisch or vice-versa.

%#############################################################
%###################### Acknowledgements #####################
%#############################################################
%\newpage
%\chapter*{Acknowledgements}

%DELETEME: Thank you for the music, the songs I am singing
    %The author expresses his gratitude towards Dipl.-Ing.\ Jo\"{e}l\ Chinnow and Dr.-Ing.\ Karsten\ Bsufka as his supervisors for their guidance and supports during the thesis. The author also wishes to thank his colleagues, Leily\ Behnam\ Sani\ and The\ Hieu\ Tran\ for their assistance and valuable ideas to the undertaking of the work summarized here.
    
    
    
