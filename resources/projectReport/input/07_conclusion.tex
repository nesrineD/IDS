\chapter{Conclusion}
\label{conclusion}

	During the course of this project all of the objectives set in chapter \ref{objectives} have been achieved and, therefore, the project goal was met.\ The authors have had first hand experience in the development of an IDS-system and how the focus-point of the IDS system can shape the architectural design which, in turn, may heavily determine the performance of the IDS system in regards of precision and recall rate.\ The missing attacks-labelling hinders the performance of the developed IDS system since the validation process requires labelled attacks in order to construct the final IDS model properly.
	
	Nevertheless the evaluation is performed on partly labelled data-set which results in succesfull identification of all attacks in the testing data-set.\ However, as described in Chapter~\ref{results}, such a high recall rate is accompanied with a low accuracy rate (high false-positives).\ Future works in this project might involve tinkering with the one-class SVM settings in order to achieve a higher accuracy rate with the minimal expense of lower recall rate.\ Additionally, introducing PCA into the IDS system properly may substantially decrease the required time to complete the training and validation process as currently the IDS system require approximately $10$ hours to perform the training and validation process.\ Lastly, a Graphical User Interface would increase the usability of the developed IDS system.