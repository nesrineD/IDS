\chapter{Introduction}
\label{introduction}

	Due to the rapidly expanding IT-infrastructure and ,thus, its complexity, IT-security has become a concern that poses increasingly difficult challenges for IT-security experts in protecting the system from malicious attacks.\ One major contributing factor is the growing complexity of cyber-attacks which, assisted by automation development, lowers the required knowledge to perform such attacks even further \cite{McHugh:IntrusionAndIntrusionDetection}.\ In order to deal with this phenomena, various cyber-defence mechanisms and tools have been developed over the years, one of which is \textit{intrusion detection system}.
	
	An intrusion detection system (IDS), proposed by Denning in 1987 \cite{Denning:IntrusionDetectionModel}, is a system that aims to detect intrusion (security-breach) into a system and fires an alert in order for security-personals to decide on necessary course of actions.\ While there has been significant amount of research in the field of IDS, it remains a young research-field and is still plagued by several weaknesses, such as high false positive rates.
	
	This project aims to tackle the task of developing an IDS system in order to be familiarised with the challenges of IDS-development as well as to be able to recognise both the strength along with the weaknesses of different IDS-strategies/-mechanisms.\ The goal of this project is to develop an IDS system that has the capability to identify malicious (anomaly) behaviour in the IT-infrastructure and provides an alert accordingly.
	
	This project report is structured as follows.\ Chapter~\ref{background} explains some necessary background that is required to understand the work of this project, Chapter~\ref{objectives} sets the objectives which are relevant in order to achieve the goal of this project; Chapter~\ref{approach} describes the approach of the authors to reach the objectives; The actual implementation details are explained in Chapter~\ref{implementation} whereas the performance of the implementation is described in Chapter~\ref{results}.\ Finally, Chapter~\ref{conclusion} sums up this project and provides some insight to future works related to this project.
